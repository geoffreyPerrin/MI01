\documentclass[11pt]{report}
\usepackage[utf8]{inputenc}
\usepackage[T1]{fontenc}
 \usepackage{listings}
\usepackage{amsmath}

\usepackage{fancyhdr}
\usepackage{graphicx}
\usepackage{color}
\pagestyle{fancy}


 \lhead{Geoffrey PERRIN - Océane DUBOIS}
 \rhead{}
 \rfoot{}




\definecolor{codegreen}{rgb}{0,0.6,0}
\definecolor{codegray}{rgb}{0.5,0.5,0.5}
\definecolor{codepurple}{rgb}{0.58,0,0.82}
\definecolor{backcolour}{rgb}{0.95,0.95,0.92}

\lstdefinestyle{mystyle}{
    backgroundcolor=\color{backcolour},
    commentstyle=\color{codegreen},
    keywordstyle=\color{magenta},
    numberstyle=\tiny\color{codegray},
    stringstyle=\color{codepurple},
    basicstyle=\footnotesize,
    breakatwhitespace=false,
    commentstyle=\color{codegreen},   
    breaklines=true,
    captionpos=b,
    keepspaces=true,
    numbers=left,
    numbersep=5pt,
    showspaces=false,
    showstringspaces=false,
    showtabs=false,
    tabsize=2,
     morekeywords={mov, push, xor, extern, div, mov, inc, cmp, jne, call, pop, ret,endp, proc, end, dec, add, jb, dd, db, jge,  not, lea, main, public, movzx, movd, psllq,imul, movq, pxor, sub, paddw, punpcklbw, pmaddwd, phaddd, shr, psrld, ja},
   extendedchars=true,
    sensitive=false,
   morecomment=[l];,
    literate= {á}{{\'a}}1 {é}{{\'e}}1 {í}{{\'i}}1 {ó}{{\'o}}1 {ú}{{\'u}}1 {Á}{{\'A}}1 {É}{{\'E}}1 {Í}{{\'I}}1 {Ó}{{\'O}}1 {Ú}{{\'U}}1 {à}{{\`a}}1 {è}{{\`e}}1 {ì}{{\`i}}1 {ò}{{\`o}}1 {ù}{{\`u}}1 {À}{{\`A}}1 {È}{{\'E}}1 {Ì}{{\`I}}1 {Ò}{{\`O}}1 {Ù}{{\`U}}1 {ä}{{\"a}}1 {ë}{{\"e}}1 {ï}{{\"i}}1 {ö}{{\"o}}1 {ü}{{\"u}}1 {Ä}{{\"A}}1 {Ë}{{\"E}}1 {Ï}{{\"I}}1 {Ö}{{\"O}}1 {Ü}{{\"U}}1 {â}{{\^a}}1 {ê}{{\^e}}1 {î}{{\^i}}1 {ô}{{\^o}}1 {û}{{\^u}}1 {Â}{{\^A}}1 {Ê}{{\^E}}1 {Î}{{\^I}}1 {Ô}{{\^O}}1 {Û}{{\^U}}1,
  }

\lstset{style=mystyle}

%Gummi|065|=)
\title{\textbf{TP08 - Traitement d'image - Troisième Partie }
\author{Geoffrey PERRIN \\ Océane DUBOIS\\}
\date{}}

\begin{document}

\maketitle

\newpage

Le but de ce TP est d'utiliser l'extension SSE pour aller plus vite lors des calculs et obtenir un programme plus condensé. 
On commence par traiter les pixels 2 par 2 en utilisant un seul registre, puis on traitera les pixel 4 par 4 en utilisant 2 registres en parallèle.

On commence donc par charger les 2 derniers pixels de l'image dans un registre, puis à l'aide de l'instruction "unpack" on sépare chaque composante RVB par mot de 0000. Ensuite on charge les coefficients multiplicateurs dans un registre, puis on les place en double dans un registre xmm2.
On unpack également xmm2, ainsi on se retrouve avec les coefficients multiplicateurs alignés avec chaque valeur RVB correspondante.  

Grâce à l'instruction pmaddd, on multiplie les coefficients avec les valeur RVB puis on les additionne 2 à 2.

Afin d'avoir les 3 résultats additionés ensemble, on complète l'instruction par un phadd. On se retrouve avec 2 fois le résultat du pixel 1 et 2 fois le résultat du pixel 2 enregistré en xmm0. 

On récupère d'abord dans eax le résultat du pixel 1, qu'on divise par 256 (car les coefficients sont à virgule fixe). Puis on place ce résultat dans le pixel de destination.
On réalise la même opération pour le pixel 2. 

On réalise ces opération tant qu'il y a des pixels à traiter. 


\begin{lstlisting}
; IMAGE_SIMD.ASM
;
; MI01 - TP Assembleur 2 à 5
;
; Réalise le traitement d'une image 32 bits.

.686
.XMM
.MODEL FLAT, C

.DATA

.CODE

; **********************************************************************
; Sous-programme _process_image_simd
;
; Réalise le traitement d'une image 32 bits avec des instructions SIMD.
;
; Entrées sur la pile : Largeur de l'image (entier 32 bits)
;           Hauteur de l'image (entier 32 bits)
;           Pointeur sur l'image source (dépl. 32 bits)
;           Pointeur sur l'image tampon 1 (dépl. 32 bits)
;           Pointeur sur l'image tampon 2 (dépl. 32 bits)
;           Pointeur sur l'image finale (dépl. 32 bits)
; **********************************************************************
PUBLIC      process_image_simd
process_image_simd   PROC NEAR       ; Point d'entrée du sous programme

        push    ebp
        mov     ebp, esp

        push    ebx
        push    esi
        push    edi

        mov     ecx, [ebp + 8]
        imul    ecx, [ebp + 12]

        mov     esi, [ebp + 16]
        mov     edi, [ebp + 20]
		sub      ecx,2

boucle: 
        ;on récupére le pixel de l'image source à traiter 
        movq xmm0, qword ptr[esi+ecx*4]
		
		
		pxor xmm3, xmm3
		punpcklbw xmm0, xmm3 


			; chargement des coeff
		
		
		mov eax, 4C961Dh ;???
		movd xmm2, eax
		psllq xmm2, 1
		movd xmm1, eax
		paddw xmm2, smm1
		punpcklbw xmm2, xmm3
		;on a 2 fois les coefficients multilicateurs

		pmaddwd  xmm0, xmm2

		phaddd xmm0, xmm0
		
		movd eax, xmm0 ;on récupère le premier coefficient
		shr eax, 8 ;on divise par 256

		add ecx, 2 ;on retroune au premier pixel
		mov [edi+ecx*4], eax

		psrld xmm0, 1

		movd eax, xmm0
		shr  eax, 8
		sub ecx, 1
		mov [edi+ecx*4], eax


		sub		ecx,1
        cmp    ecx, 0
        ja     boucle

fin:
        pop     edi
        pop     esi
        pop     ebx

        pop     ebp

        ret                         ; Retour à la fonction MainWndProc

process_image_simd   ENDP
END

\end{lstlisting}


TEMPS !!!!!


Dans un deuxième temps, on souhaite traiter les pixels 4 par 4, pour cela on réalise les mêmes opérations mais en utilisant 2 registres xmm en même temps. 

Voici le code :

\begin{lstlisting}
        push    ebp
        mov     ebp, esp

        push    ebx
        push    esi
        push    edi

        mov     ecx, [ebp + 8]
        imul    ecx, [ebp + 12]

        mov     esi, [ebp + 16]
        mov     edi, [ebp + 20]
		sub      ecx,2

boucle: 
        ;on récupére le pixel de l'image source à traiter 
        movq xmm0, qword ptr[esi+ecx*4]
	sub ecx, 2
	movq xmm1, qword ptr[esi+ecx*4]
		
		
	pxor xmm3, xmm3
	punpcklbw xmm0, xmm3 
	punpcklbw xmm1, xmm3 


	; chargement des coeff
		
	mov eax, 4C961Dh ;???
	movd xmm2, eax
	psllq xmm2, 1
	movd xmm1, eax
	paddw xmm2, smm1
	punpcklbw xmm2, xmm3
	;on a 2 fois les coefficients multilicateurs

	pmaddwd  xmm0, xmm2
	pmaddwd  xmm1, xmm2

	phaddd xmm0, xmm0
	phaddd xmm1, xmm1
		
	movd eax, xmm0 ;on récupère le premier coefficient
	shr eax, 8 ;on divise par 256

	add ecx, 4 ;on retroune au premier pixel
	mov [edi+ecx*4], eax

	psrld xmm0, 1

	movd eax, xmm0
	shr  eax, 8
	sub ecx, 1
	mov [edi+ecx*4], eax

	movd eax, xmm1 ;on récupère le premier coefficient
	shr eax, 8 ;on divise par 256

	sub ecx, 1
	mov [edi+ecx*4], eax

	psrld xmm1, 1

	movd eax, xmm1
	shr  eax, 8
	sub ecx, 1
	mov [edi+ecx*4], eax

	sub	ecx,4
        cmp    ecx, 0
        ja     boucle

fin:
        pop     edi
        pop     esi
        pop     ebx

        pop     ebp

        ret       
\end{lstlisting}





\end{document}